\newif\ifbsixpaper
\bsixpapertrue
\ifbsixpaper
\documentclass{book}
\usepackage[
  b6paper,
  paperheight=7.367in,
  hmargin=.4in,
  top=.55in,
  bottom=.5in,
  headsep=1.5ex,
  footskip=3ex
]{geometry}
\else
\documentclass[a4paper]{book}
\usepackage[hmargin=1.5in,vmargin=1in]{geometry}
\fi

\newcommand*{\titleContent}{Symbolic Logic}
\newcommand*{\subtitleContent}{An Introduction}
\newcommand*{\authorContent}{Frederic Breton Fitch}
\newcommand*{\publisherContent}{The Ronald Press Company}

\title{\titleContent: \subtitleContent}
\author{\authorContent}
\date{}

\usepackage[pagestyles]{titlesec}

\titleclass{\chapter}{top}
\titleformat{\chapter}[display]{\centering\large}{%
  Chapter \thechapter}{1ex}{\normalsize\MakeUppercase}
\titlespacing{\chapter}{0pt}{40pt}{10pt}
\titleformat{\section}[block]{\centering\bfseries}{\thesection.}{1ex}{}
\titleformat{name=\section,numberless}[block]{%
  \centering\bfseries}{}{0ex}{\small\MakeUppercase}
\titlespacing{\section}{0pt}{*1}{1ex minus 1ex}
\titlespacing{name=\section,numberless}{0pt}{*2}{1ex minus 1ex}
\titleformat{\subsection}[runin]{\bfseries}{\indent\thesubsection.}{0pt}{}
\titlespacing{\subsection}{0pt}{0pt plus 3pt}{1ex}

\newpagestyle{fmatter}[\footnotesize]{
  \sethead[\thepage][\MakeUppercase{\chaptertitle}][]
  {}{\chaptertitle}{\thepage}}
\renewpagestyle{headings}[\footnotesize]{
  \sethead[\thepage][\MakeUppercase{\titleContent}][\lbrack Ch.\ \thechapter]
  {Ch.\ \thechapter\rbrack}{\chaptertitle}{\thepage}}

\usepackage{chngcntr}
\counterwithout{section}{chapter}

\usepackage{enumitem}
\newlist{abcd}{enumerate}{1}
\setlist[abcd]{label=\Alph*.,left=2\parindent,nosep,topsep=3pt}
\setlist[enumerate,1]{left=0pt,nosep}
\setlist[enumerate,2]{nosep}

% https://www.logicmatters.net/resources/fitch.sty
% https://www.logicmatters.net/resources/fitchguide.txt
\usepackage{fitch}
\setlength{\fitchnumwd}{2pt}
% \setlength{\fitchindent}{0pt}

\usepackage{xcolor}
\usepackage[
  pdfusetitle,
  hyperfootnotes=false,
  colorlinks=true,
  urlcolor={.},
  linkcolor={.}
]{hyperref}

\renewcommand*{\neg}{\mathord{\sim}}

\interfootnotelinepenalty=10000

\hyphenation{Physicalisch=mathe-matische}

% \let\oldtextbf\textbf
% \newcommand*{\textBF}[1]{\sbox0{#1}\resizebox{\wd0}{\ht0}{\oldtextbf{#1}}}
% \let\textbf\textBF

\makeatletter
\let\openrightfalse\@openrightfalse
\makeatother

\AtBeginDocument{
  \setlength{\parindent}{10pt}
  \setlength{\parskip}{0pt plus 3pt}
}

\begin{document}
\frontmatter
\pagestyle{fmatter}

\currentpdfbookmark{Title Page}{titlepage}
\begin{titlepage}
  \vspace*{\stretch{1}}
  \begin{center}
    \MakeUppercase{
      \Huge\bfseries\titleContent} \\[2em]
      \huge\itshape\subtitleContent
    \end{center}
  \vspace*{\stretch{1}}
  \begin{center}
    By \\[1em]
    \MakeUppercase{
      \authorContent \\[1em]
      \tiny Professor of Philosophy \\
      Yale University}
  \end{center}
  \vspace*{\stretch{5}}
  \begin{center}
    \MakeUppercase{\publisherContent\ · New York}
  \end{center}
  \vspace*{\stretch{.5}}
\end{titlepage}

\currentpdfbookmark{Copyright Page}{copyright}
\thispagestyle{empty}
\vspace*{\stretch{3}}
\begin{center}
  Copyright, 1952, by \\[1ex]
  \textsc{\publisherContent} \\[1ex]
  ———\\[1ex]
  \textit{All Rights Reserved} \\[1em]
  \begin{minipage}{2in}
    \footnotesize The text of this publication or any part thereof may not be reproduced in any manner whatsoever without permission in writing from the publisher.
  \end{minipage}

  \vspace*{\stretch{1}}

  3 \\
  \textsc{vr-vr}
\end{center}
\vspace*{\stretch{5}}
\begin{center}
  \small Library of Congress Catalog Card Number: 52-6196 \\
  \MakeUppercase{\tiny Printed in the United States of America}
\end{center}
% \vspace*{\stretch{.2}}

\clearpage
\currentpdfbookmark{Preface}{preface}
\chapter*{\vspace*{3ex}Preface}
\chaptermark{Preface}

This book is intended both as a textbook in symbolic logic for undergraduate and graduate students and as a treatise on the foundations of logic.  Much of the material was developed in an undergraduate course given for some years in Yale University.  The course was essentially a first course in logic for students interested in science.  Many alternative devices and methods of presentation were tried.  Those included here are the ones that seemed most successful.

The early sections of the book present rules for working with implication, conjunction, disjunction, and negation.  In connection with negation, there is a discussion of Heyting's system of logic and the law of excluded middle.  Following this, various modal concepts such as necessity, possibility, and strict implication are introduced.  The theory of identity and the general theory of classes and relations are presented.  The theory of quantifiers is then developed.  Finally, operations on classes and relations are defined and discussed.  The book provides a novel way for avoiding Russell's paradox and other similar paradoxes.  No theory of types is required.  The system of logic employed is shown to be free from contradiction.  There are three appendices: Appendix A shows how classes can be defined by means of four operators using techniques similar to those of Curry's combinatory logic.  Appendix B shows in outline how the system can be further extended so as to form a consistent foundation for a large part of mathematics.  Appendix C discusses an important kind of philosophical reasoning and indicates why the system of logic of the book is especially well suited for handling it.

The student not acquainted with symbolic logic can omit Sections 20 and 27 which are of a more difficult nature than the other sections.  The three appendices are also of a somewhat advanced nature.  These appendices and the Foreword are addressed mainly to readers who already have some knowledge of symbolic logic.

The sections concerned with modal logic, namely, 11, 12, 13, and 23, can be omitted if desired, since the other sections do not depend essentially on them.

I am greatly indebted to my past teachers and to my present colleagues and students for their inspiring interest in logic and philosophy and for their helpful insights and suggestions.  I am also, of course, greatly indebted to many contemporary writers in logic and allied fields.  In some ways my debt is greatest to Professor Filmer S. C. Northrop, since he made clear to me the importance of modern logic and guided my first work in it.  Some of the fundamental ideas of this system of logic were conceived during the tenure of a John Simon Guggenheim Memorial Fellowship in 1945–1946.

Thanks are due to Miss Erna F. Schneider for her careful reading of a large part of the manuscript.  She made numerous useful suggestions.  I am also very grateful to Dr.\ John R. Myhill for studying some of the more difficult portions of the manuscript, for pointing out some logical and typographical errors, and for making various constructive criticisms.  I wish to thank Mr.\ I. Sussmann for calling my attention to typographical errors, and Miss Mabel R. Weld for her help in typing the manuscript.  Various members of the Yale philosophy department also made helpful suggestions regarding the general scheme of the book.  I am indebted to Mr.\ Herbert P. Galliher and Mr.\ Abner E. Shimony for valuable comments on the manuscript.

\vspace{1em}
\hfill\textsc{Frederic B. Fitch}

\noindent New Haven, Conn.

February, 1952

\clearpage
\currentpdfbookmark{Foreword}{foreword}
\chapter*{Foreword}
\chaptermark{Foreword}

Five outstanding characteristics of the system of logic of this book are as follows:

(1) It is a system that can be proved free from contradiction, so there is no danger of any of the standard logical paradoxes arising in it, such as Russell's paradox or Burali-Forti's paradox.  In Sections 20 and 27 a proof will be given of the consistency of as much of the system as is presented in the present volume.  In Appendix B a proof of the consistency of the rest of the system is outlined.

(2) The system seems to be adequate for all of mathematics essential to the natural sciences.  The main principles of mathematical analysis will be derived in a subsequent volume.  Apparently no other system of logic, adequate for as large a portion of standard mathematics, is now known to be free from contradiction.

(3) The system is not encumbered by any “theory of types”.  The disadvantage of a theory of types is that it treats as “meaningless” all propositions that are concerned with attributes or classes in general.  A logic with a theory of types is of little or no use in philosophy, since philosophy must be free to make completely general statements about attributes and classes.  A theory of types also has the disadvantage of ruling out as “meaningless” some philosophically important types of argument which involve propositions that have the character of referring directly or indirectly to themselves.  In Appendix C there is a discussion of the nature and importance of these self-referential propositions.  Furthermore, a theory of types, if viewed as applying to all classes, cannot itself even be stated without violating its own principles.  Such a statement would be concerned with all classes and so would be meaningless according to the principles of such a theory of types itself.  This point has been made by Paul Weiss\footnote{Paul Weiss, “The Theory of Types”, Mind, n.s., vol.\ 37 (1928), pp.\ 338–48.} and myself\footnote{F. B. Fitch, “Self-Reference in Philosophy”, Mind, n.s., vol.\ 55 (1946), pp.\ 64–73.  This article is reprinted in Appendix C.}.

(4) The system employs the “method of subordinate proofs”, a method that vastly simplifies the carrying out of complicated proofs and that enables the reader to gain rapidly a real sense of mastery of symbolic logic.

(5) The system is a “modal logic”; that is, it deals not only with the usual concepts of logic, such as conjunction, disjunction, negation, abstraction, and quantification, but also with logical necessity and logical possibility.

No great stress is laid on the contrast between syntax and semantics, or on the finer points concerning the semantical use of quotation marks.  The reason for this is that such emphasis very often produces unnecessary difficulties in the mind of a person first approaching the subject of symbolic logic, and inhibits his ability to perform the fundamental operations with ease.  The use of quotation marks will be found to be rather informal.  This is done deliberately for pedagogical convenience.  The semantical paradoxes, incidentally, are avoided by this system of logic in the same way that it avoids the purely logical and mathematical paradoxes.

Numerous exercises have been provided. Even the logically sophisticated reader will get a better understanding of the material by doing some of the exercises.

In comparing this system with some other well-known systems, it can be said to appear to be superior to the Whitehead-Russell system,\footnote{A. N. Whitehead and Bertrand Russell, \textit{Principia Mathematica}, 3 vols., Cambridge, England, 1910, 1912, 1913. Second edition, 1925, 1927. Reprinted 1950.} at least with respect to its demonstrable consistency and its freedom from a theory of types.  In place of Russell's “vicious circle principle”\footnote{\textit{Principia Mathematica}, Chapter II of the Introduction to the first edition.} for avoiding paradoxes, my system uses a weakened law of excluded middle (see 10.16 and 10.19) and the following novel principle: A proposition \(p\) is not to be regarded as validly proved by a proof that makes essential use of the fact that some proposition other than \(p\) follows logically from \(p\).  This principle is reminiscent of the classical prohibition against treating a proposition as true simply because it implies another proposition that is true.  My principle, however, makes no stipulation about the truth or falsity of the proposition implied by p.  (See 18.6 for a technical statement of this principle.)

The demonstrable consistency of the present system is also an important point in its favor when compared with the “set-theoretical” logics proposed by Zermelo,\footnote{Ernst Zermelo, “Untersuchungen über die Grundlagen der Mengenlehre I”, \textit{Math.\ Annalen}, vol.\ 65 (1908), pp.\ 261–81.} Fraenkel,\footnote{Adolf Fraenkel, “Untersuchungen über die Grundlagen der Mengenlehre”, \textit{Math.\ Z.}, vol.\ 22 (1925), pp.\ 250–73.} von Neumann,\footnote{J. von Neumann, “Eine Axiomatisierung der Mengenlehre”, \textit{Jour.\ r.\ angew.\ Math.}, vol.\ 154 (1925), pp.\ 219–40; “Die Axiomatisierung der Mengenlehre”, \textit{Math.\ Z.}, vol.\ 27 (1928), pp.\ 669–752.} and Bernays.\footnote{Paul Bernays, “A System of Axiomatic Set Theory”, \textit{Journal of Symbolic Logic}, vol.\ 2 (1937), pp.\ 65–77; vol.\ 6 (1941), pp.\ 1–17; vol.\ 7 (1942), pp.\ 65–89, 133–45; vol.\ 8 (1943), pp.\ 89–106; vol.\ 13 (1948), pp.\ 65–79.}  These logics are widely used by mathematicians but are not known to be free from contradiction.  Even these systems, in their most fully developed forms, employ what is almost a theory of types.  For example, in Gödel's reformulation\footnote{Kurt Gödel, \textit{The Consistency of the Continuum Hypothesis}, Princeton, 1940.} of the Bernays system, the distinction between “set”, “class”, and “notion” is very much like the distinction between three successively higher “types” or “orders”.

Quine\footnote{For example, the system of W. V. Quine's book, \textit{Mathematical Logic} (New York, 1940).  See also his paper, “New Foundations for Mathematical Logic”, \textit{American Mathematical Monthly}, vol.\ 44 (1937), pp.\ 70–80.  For a discussion of Quine's systems in connection with problems of consistency, see Hao Wang, “A Formal System of Logic”, \textit{Journal of Symbolic Logic}, vol.\ 15 (1950), pp.\ 25–32.} has constructed various interesting and elegant systems that bear close affiliations with the Whitehead-Russell logic and with the set-theoretic logics just mentioned.

All these systems with which the present system is being compared are characterized by the fact that none of them permits the formation of attributes or classes with the ease and freedom allowed by the rules stated in Section 17.  (See in particular 17.4 and 17.5.)  The restrictions imposed on this freedom by other systems seem arbitrary and philosophically unconvincing.

The method of subordinate proofs was suggested by techniques due to Gentzen\footnote{Gerhard Gentzen, “Untersuchungen über das logische Schliessen”, \textit{Math.\ Z.}, vol.\ 39 (1934), pp.\ 176–210, 405–31.} and Jaśkowski.\footnote{Stanislaw Jaśkowski, “On the Rules of Suppositions in Formal Logic”, \textit{Studia Logica}, No.\ 1, Warsaw, 1934.}  It has various pedagogical advantages and also facilitates comparison of the theory of negation of this book with the theory of negation of the intuitionistic logic of Heyting,\footnote{A. Heyting, “Die formalen Regeln der intuitionistischen Logik”, \textit{Sitzungsberichte der Preussischen Akademie der Wissenschaften} (Physicalisch-mathematische Klasse), 1930, pp.\ 42–56.  See also, \textit{ibid.}, pp.\ 57–71, 158–69.} as is shown in Section 10.  This method has been used in my teaching for the past eleven years.

The treatment of modality is very similar to that employed by Lewis and Langford\footnote{C. I. Lewis and C. H. Langford, \textit{Symbolic Logic}, New York, 1932.} and by Ruth Barcan Marcus,\footnote{Ruth C. Barcan (Ruth Barcan Marcus), “A Functional Calculus of First Order Based on Strict Implication”, \textit{Journal of Symbolic Logic}, vol.\ 11 (1946), pp.\ 1–16.  See also, \textit{ibid.}, pp.\ 115–18; vol.\ 12 (1947), pp.\ 12–15.} but the subordinate proof technique in this connection is an innovation.

The system of this book is closely similar to, and in certain important respects an improvement on, the system of my paper, “An Extension of Basic Logic”,\footnote{\textit{Journal of Symbolic Logic}, vol.\ 13 (1948), pp.\ 95–106.} and that of my paper, “A Further Consistent Extension of Basic Logic”.\footnote{\textit{Ibid.}, vol.\ 14, No.\ 4 (1950), pp.\ 209–18.}  The improvement over both the latter systems consists in a more adequate theory of implication and the restriction of all proofs to finite length, so that the resulting logic can be said to be “finitary”.  Analogues of the rules [\(\#\)], [\(\neg\#\)], [\(*\)], and [\(\neg *\)] of 3.1 of “An Extension of Basic Logic” (and of the rules 3.28 and 3.29 of the other paper) are not given in the main body of this volume, but they are stated in Appendix B as rules R41–R44.  These rules together with R45 complete the total of forty-five rules needed for formulating the whole system.  An outline is given in Appendix B of a consistency proof for this total system.  A derivation of the more important theorems of mathematical analysis from these forty-five rules can proceed along the lines of my papers, “The Heine-Borel Theorem in Extended Basic Logic”\footnote{\textit{Ibid.}, vol.\ 14, No.\ 1 (1949), pp.\ 9–15.} and “A Demonstrably Consistent Mathematics”.\footnote{\textit{Ibid.}, vol.\ 15, No.\ 1 (1950), pp.\ 17–24; vol.\ 16, No.\ 2 (1951), pp.\ 121–4.}  This will be done in detail in a subsequent volume.  The systems of logic of the two papers just cited are non-finitary, while the system based on the forty-five rules has the asset of being finitary.

\setcounter{tocdepth}{1}
\tableofcontents

\mainmatter
\pagestyle{headings}

\vspace*{\stretch{1}}
\begin{center}
  \Large\bfseries\MakeUppercase{Symbolic Logic}
\end{center}
\vspace*{\stretch{2}}

\thispagestyle{empty}
\clearpage
\thispagestyle{empty}
\cleardoublepage

\chapter{Symbolic Logic and Formal Proofs}
\label{chap:1}

\section{Introduction}
\label{sec:1}

\subsection{}
\label{sec:1.1}

Modern deductive logic, also known as “symbolic logic” or “mathematical logic”, arose in the nineteenth century from earlier systems of logic, especially Aristotelian logic, and from traditional mathematics.  In a sense it now embraces all these sources from which it came.  The Aristotelian forms of inference appear within it as special cases of more general forms of reasoning, while the laws of mathematics are likewise derivable within it.  Symbolic logic also represents an important advance beyond Aristotelian logic and ordinary mathematics.  It surpasses the former in being able to deal far more adequately with complicated relational structures.  It surpasses the latter in being able to deal more powerfully with non-quantitative concepts.  The non-quantitative concepts handled by mathematics tend to be fairly direct generalizations from quantitative concepts, while symbolic logic can deal in addition with non-quantitative concepts that are not generalizations of this sort.

\subsection{}
\label{sec:1.2}

No satisfactory theory of relations is provided by Aristotelian logic.  The only relations that were even partly amenable to genuine logical treatment in past centuries were the familiar numerical and geometrical relations of mathematics, together with the relation of identity and some relations of implication and class-inclusion.

\subsection{}
\label{sec:1.3}

When philosophers of the past attempted to exploit this fact that mathematical relations were, for them, almost the only relations that could be handled with logical precision, the tendency was toward a quantitative view of the universe.  The emphasis was on geometry or on materialism, or on both.  Often there was a deprecation of the unmeasurable qualitative and aesthetic factors in the world, factors, which are the special concern of art, literature, and religion.  One way of deprecating them was to say that they were “merely in the mind” and hence “subjective” and “unreal”.  This did not actually dispose of them however, because the mind and its contents are themselves part of the universe.

\subsection{}
\label{sec:1.4}

When identity and implication, rather than mathematical relations, were the relations most emphasized by philosophers, and when relatedness was treated as disguised or partial identity or implication, the tendency was toward monism (“All is one”) and toward an exaggeration of the importance of the whole of the universe at the expense of the parts.  Implication itself was often viewed as a kind of partial identity, so that one thing implied another if the latter was identical with part of the former.  Thus all relatedness reduced to identity, and the final identity was identity with the final one reality.  Differences tended to be treated as illusory, but there was always the residual problem of how there could be even illusory differences and how these illusory differences could be related to each other and to the one reality without becoming identical with each other and with the one reality.  If such a philosophy becomes politically influential, its outcome is likely to be totalitarianism.  The totalitarianism of Hitler can perhaps be regarded as derived in some degree from Hegel's monism.  The totalitarianism of Stalin is clearly derived from Hegel by way of Marx, who added an element of materialism to Hegel's philosophy.

\subsection{}
\label{sec:1.5}

When no relations were particularly emphasized in philosophic thought, or at least none that could be easily handled by logical procedures then available, the tendency was toward a pluralism of loosely related “substances”, sometimes with God or “mind” or “pre-established harmony” serving as a relating factor.  Such a view, though deficient in some respects, at least would allow for the non-mathematical aspects of the world and would not tend to treat all differences as illusory and all political differences as undesirable.

\subsection{}
\label{sec:1.6}

These various types of philosophy, when properly and sympathetically understood, are perhaps less divergent from one another and less defective than the inadequate systems of logic, on which they had to depend, forced them to appear to be.  With the extraordinary development of logic during the first half of the twentieth century, mankind for the first time finds itself in possession of a tool that is powerful enough to be of help in reasoning about relations and qualities of all sorts.  There have already been applications of symbolic logic to problems in biology,\footnote{J. H. Woodger, \textit{The Axiomatic Method in Biology} (Cambridge: Cambridge University Press, 1937).  Also, “Technique of Theory Construction”, International Encyclopedia of Unified Science (Chicago, 1939).} neurophysiology,\footnote{W. S. McCulloch and W. Pitts, “A Logical Calculus of the Ideas Immanent in Nervous Activity”, \textit{Bulletin of Mathematical Biophysics}, vol.\ 5 (1943), pp.\ 115–33.} engineering,\footnote{C. E. Shannon, “A Symbolic Analysis of Relay and Switching Circuits”, \textit{Transactions of the American Institute of Electrical Engineers}, vol.\ 57 (1938), pp.\ 713–23.} psychology,\footnote{C. L. Hull, C. I. Hovland, \textit{et al.}, \textit{Mathematico-Dedudive Theory of Rote Learning} (New Haven, 1940).} and philosophy.\footnote{W. V. Quine, “On What There Is”, \textit{Review of Metaphysics}, vol.\ 2 (1948), pp.\ 21–38.  Also, F. B. Fitch, “Actuality, Possibility, and Being”, \textit{ibid.}, vol.\ 3 (1949), pp.\ 367–84.  The writings of Rudolf Carnap should also be mentioned; for example, \textit{Meaning and Necessity} (Chicago, 1947).}  Some day it may be possible for experts in symbolic logic to think as clearly and as effectively about social, moral, and aesthetic concepts as experts in mathematics have long been able to do with respect to the “colorless” ideas of physics.  The full impact of the new science of logic has not yet been felt.  This is partly because its theoretical development is not yet complete, and partly because it has not yet been learned by many who could most profitably use it.  When its impact is felt, a richer, more human, and more rational philosophy may gradually arise.  The day may come when it will be as improper to study ethics and politics without a thorough grounding in symbolic logic as it now is to study physics without a thorough grounding in mathematics.  Man's ingenuity in devising a workable system of world peace may then have a chance of equaling his ingenuity in devising the atomic bomb.

\subsection{}
\label{sec:1.7}

If the pages of this book seem rather technical, so are the pages of any reputable book on mathematics or physics.  Philosophy is no less technical than physics.  It deals with even more of the world than physics does; in fact, with everything.  We should not expect the logic suitable for philosophy, ethics, and aesthetics to be very much less technical than the mathematics suitable for physics and chemistry.

\section{The Nature of Propositions}
\label{sec:2}

\subsection{}
\label{sec:2.1}

Certain combinations of words constitute word groups called “sentences”.

\subsection{}
\label{sec:2.2}

Every sentence has one or more “meanings”, depending on how it is interpreted.  Thus the sentence, “He can put two and two together”, obviously has more than one meaning.  We shall restrict our attention to sentences which have only one meaning, or at least we shall assume that there is always a preferred or intended meaning which we shall call “the meaning” of the sentence.  We will not attempt to discuss here the difficult problem of how meaning is communicated.

\subsection{}
\label{sec:2.3}

Meanings of sentences may also be called “verbalized propositions”.  Every verbalized proposition is the meaning of some sentence.  Roughly speaking, a proposition is anything that might conceivably be the meaning of some sentence, whether or not the requisite sentence has ever been formulated or uttered, and hence whether the proposition is verbalized or not.  We often have vague feelings or premonitions that we cannot easily express in words.  These are unverbalized propositions.

\subsection{}
\label{sec:2.4}

Some sentences are true and others are false.  The meaning of a true sentence is a true proposition, and the meaning of a false sentence is a false proposition.  The meaning of the sentence, “The earth revolves around the sun”, is the true proposition that the earth does revolve around the sun.  The sentence, “Hydrogen is heavier than oxygen”, has as its meaning a false proposition.

\subsection{}
\label{sec:2.5}

Propositions can be objects of belief and disbelief.  Thus someone may believe the false proposition expressed by the sentence, “The earth is flat”, while someone else may disbelieve this same proposition.

\subsection{}
\label{sec:2.6}

Sentences are usually referred to by writing them in quotation marks, as has been done above.  A proposition may be referred to by first writing in quotation marks a sentence that means the proposition, and by then referring to the proposition as the meaning of the quoted sentence.  A second method is to use a subordinate noun clause expressing the meaning of the sentence.  According to the first method, we would speak of the proposition expressed by the sentence, “The earth is flat”.  According to the second method, we would speak of the proposition that the earth is flat.  Both methods are used in paragraph \ref{sec:2.4}.  Similarly, we might say that Columbus did not believe the proposition expressed by the sentence, “The earth is flat”, or we might equally well say that Columbus did not believe the proposition that the earth is flat.  Still more briefly, we could say that Columbus did not believe that the earth is flat.  A still different method is to say that Columbus did not believe the proposition, “The earth is flat”.  This is simply a shorthand way of saying that Columbus did not believe the proposition expressed by the sentence, “The earth is flat”.

\subsection{}
\label{sec:2.7}

True propositions may also be called “facts” or “truths”.  Thus it is a true proposition that the earth is not flat, and it is also a truth and a fact that the earth is not flat.

\subsection{}
\label{sec:2.8}

False propositions may be called “untruths” or “counterfacts”.  It is an untruth that the earth is flat.  Similarly, it is an untruth that \(4 + 4 = 9\), while it is a fact (or truth) that \(4 + 4 = 8\).

\subsection{}
\label{sec:2.9}

There are two important kinds of facts (or truths): contingent facts and non-contingent facts.  Contingent facts are true without being true by logical necessity, while non-contingent facts are true by logical necessity.  Thus it is true that men discovered the usefulness of fire, but this discovery was the outcome of practical necessity and not of logical necessity.  Indeed, some savage tribes are said not to have made this discovery yet.  So it is only a contingent fact that men discovered the usefulness of fire.  On the other hand it is true by logical necessity that \(1 + 1 = 2\), so \(1 + 1 = 2\) is a non-contingent fact.  Logic and mathematics are concerned mainly with non-contingent facts.  The other special sciences and arts deal mainly with contingent facts.  Philosophy deals with both kinds of facts in their interrelationships.

\subsection{}
\label{sec:2.10}

Some thinkers have advocated the view that all facts are contingent and that what appear to be non-contingent, logically necessary truths, are merely arbitrary conventions about the use of symbols, or are somehow merely the outcome of such conventions.  Thus they might say that \(2 + 2 = 4\) is necessarily true only because, and only in the sense that, we have agreed by an arbitrary convention to use the symbols “\(2\)”, “\(+\)”, “\(=\)”, and “\(4\)” in this way.  The only necessity they would find in such an equation would be the necessity of accepting conventions we have agreed to accept.  This view has never been developed very satisfactorily, or in sufficient detail, in my opinion.  For the purposes of this book, therefore, I shall continue to adhere to my own view that there are non-contingent truths as well as contingent truths, and that non-contingent truths are something other than the outcome of mere conventions.

\subsection{}
\label{sec:2.11}

We will also assume that there are propositions which, by logical necessity, are false, for example, \(2 + 2 = 5\).  These propositions are non-contingent untruths.

\subsection{}
\label{sec:2.12}

Furthermore, we shall assume that there are some propositions which are not to be asserted as true or false.  Examples of these will be given later.  They will be called “indefinite” propositions.  Propositions which are true or false will be called “definite” propositions.  The “principle of excluded middle” asserts that all proposi tions are true or false.  This principle will not be asserted here except in the limited sense of applying to definite propositions.

\subsection{}
\label{sec:2.13}

The classification of propositions as so far given may be outlined as follows:
\begin{abcd}
\item Definite.
  \begin{enumerate}
  \item True.
    \begin{enumerate}
    \item Necessarily true.
    \item Contingently true.
    \end{enumerate}
  \item False.
    \begin{enumerate}
    \item Necessarily false.
    \item Contingently false.
    \end{enumerate}
  \end{enumerate}
\item Indefinite.
\end{abcd}

\subsection{}
\label{sec:2.14}

An example of an indefinite proposition is the proposition expressed by the sentence, “This proposition itself is false”.  The latter proposition cannot be assumed true without also being assumed false, nor can it be assumed false without also being assumed true.  If such a proposition is regarded as satisfying the principle of excluded middle, then it must be treated as either true or false, and hence as both true and false.  So we do not assert that it satisfies the principle of excluded middle.

\subsection{}
\label{sec:2.15}

Propositions are not to be thought of as located in space and time.  Consider, for example, the true proposition, or fact, that the earth revolves around the sun.  The sun has a location in space and time, and the earth has a location in space and time, but the fact that the earth revolves around the sun does not have any genuine location in space and time.  If we were to try to assign this fact to some specific region of space time, the exact limits of such a region would be impossible to specify.  Similarly, the fact that grass is green is not located anywhere, though grass itself and other green things do have location.  Just as facts or truths have no space-time location, so also counterfacts or untruths have no location in space and time.  The untruth that \(2 + 2 = 5\) has no more location than the truth that \(2 + 2 = 4\).  It is a mistake to argue that because propositions have no location in space and time there can be no such things as propositions.  The word “thing” does not have to be used in such a narrow way as to mean only “thing located in space time”.  Numbers themselves are in a sense “things” but are not located anywhere.  Marks standing for numbers may have location, as on a blackboard, and measurements involving numbers may characterize located objects.

\subsection{}
\label{sec:2.16}

Propositions, finally, are not to be thought of as “merely mental” things.  The fact that the earth goes around the sun is not just a mental thing.  In other words, propositions are no more “located in the mind” than they are located in space and time.  But the mind may be in relationship to various propositions, as when it believes or disbelieves them.  The mind may also be in relationship to various objects that do have space-time location.

\section{Complex If-Then Propositions}
\label{sec:3}

\subsection{}
\label{sec:3.1}

Given any two propositions, we may combine them into a compound proposition by use of such conjunctions as “and”, “or”, “but”.  Thus the proposition that roses are red and the proposition that violets are blue may be combined to give the proposition that roses are red and violets are blue.  In general, if we let “\(p\)” and “\(q\)” stand for any two propositions, then “\(p\) and \(q\)” can stand for the proposition expressed by the sentence that results from writing “and” between the sentences expressing these two propositions, and “\(p\) or \(q\)” can stand for the proposition expressed by the sentence that results from writing “or” between the sentences expressing these two propositions, and so on for other conjunctions or conjunctive phrases such as “but”, “if”, “only if”, “if and only if”, “so”, “therefore”, and others.  For greater clarity the compound proposition “\(p\) and \(q\)” will hereafter be written with square brackets as follows: [\(p\) and \(q\)].  Similarly, “\(p\) or \(q\)” will be written, [\(p\) or \(q\)].  (See 6.1 and 8.1.)

\subsection{}
\label{sec:3.2}

Another very important conjunctive phrase is the phrase, “if~.~.~. then .~.~.\,”\,.  The triple dots indicate where the two sentences are to be placed which the phrase can serve to join.  Thus from the proposition, “Roses are red”, and the proposition, “Violets are blue”, we may form the compound proposition, “If roses are red then violets are blue”.  This resulting proposition can be written, [If (roses are red) then (violets are blue)].  The parentheses merely serve to set off the smaller propositions out of which the larger proposition is compounded.  Notice that we are simply considering methods of compounding propositions into larger and more complex propositions, regardless of the truth or falsity of the larger propositions thus obtained.  Here are some further examples of propositions that have been compounded out of smaller propositions:
\begin{description}[left=2\parindent,itemindent=-2ex,labelsep=0pt, nosep,topsep=3pt]
\item If one and one make two, then one and one make three.
\item If Socrates is a man and all men are mortal, then Socrates is mortal.
\item If Jack loves Jill, then Jill loves Jack.
\item If F. D. Roosevelt is elected, then Henry agrees to eat his own hat.
\end{description}

\subsection{}
\label{sec:3.3}

For purposes of symbolic logic it is desirable to build up more and more complex propositions from relatively simple ones by repeated use of the if-then phrase.  In so doing, the simplest propositions will sometimes be set off by use of parentheses as was done in an example in the above paragraph, while compound propositions will be set off by use of square brackets.  Thus the second proposition in the above list could be written, [If [(Socrates is a man) and (all men are mortal)] then (Socrates is mortal)].  This is doubly compound.  It is of the general form, [If [p and q] then r].  Unless there is some indication to the contrary, the letters “\(p\)”, “\(q\)”, “\(r\)”, “\(s\)”, and “\(t\)” are always hereafter to be thought of as standing for propositions.

\subsection{}
\label{sec:3.4}

Propositions built up by repeated use of various conjunctive phrases, including the if-then phrase, are often clumsy and inelegant from a literary standpoint, but they may be very useful tools in logical analysis.  The following sentence, though somewhat clumsy, expresses not only a genuine proposition but even a necessarily true one: “If Jack loves Jill, then, if Jill loves Tom, Jack loves Jill”.  In our special symbolism this proposition could be written, [If (Jack loves Jill) then [if (Jill loves Tom) then (Jack loves Jill)]].  The proposition can be stated in a more colloquial form somewhat as follows: “If Jack loves Jill, then even if Jill loves Tom, it is still true that Jack loves Jill”.  This proposition is of the general form, [If \(p\) then [if \(q\) then \(p\)]].  Any compound proposition of this form is a necessarily true proposition.  This is simply the principle that if a proposition is true, it is true regardless of what else may be the case.  Granted that \(p\) is true, then \(p\) is also true under the condition that \(q\) is true, no matter what the condition \(q\) may be.

\subsection{}
\label{sec:3.5}

Accordingly any proposition of the form, [If \(p\) then [if \(q\) then \(p\)]], will be called an axiom, or more specifically, an \textbf{axiom of conditioned repetition}.  Every proposition which is an axiom of conditioned repetition is necessarily true.

\subsection{}
\label{sec:3.6}

If \(p\) and \(q\) are propositions, we will say that the proposition \(q\) is a \textbf{direct consequence} of the pair of propositions,
\begin{enumerate}[label=(\arabic*),left=2\parindent,topsep=1ex,itemsep=1pt]
\item \label{3.item:1} \(p\),
\item \label{3.item:2} [If \(p\) then \(q\)].
\end{enumerate}
More specifically, we will say that it is a direct consequence by the \textbf{rule of modus ponens}.  Clearly, if propositions \ref{3.item:1} and \ref{3.item:2} are both true, then \(q\) itself is true.  Furthermore, if \ref{3.item:1} and \ref{3.item:2} are both necessarily true, and not merely contingently true, then \(q\) must also be a necessary truth.  For example, the propositions \ref{3.item:3} and \ref{3.item:4} below are necessarily true.  Therefore \ref{3.item:5}, which is a direct consequence of them by modus ponens, is also a necessarily true proposition.
\begin{enumerate}[resume*]
\item \label{3.item:3} \(a + b = b + a\).
\item \label{3.item:4} If \(a + b = b + a\), then \(c + (a + b) = c + (b + a)\).
\item \label{3.item:5} \(c + (a + b) = c + (b + a)\).
\end{enumerate}

\subsection{}
\label{sec:3.7}

We regard modus ponens as a \textbf{rule of direct consequence} according to which one proposition is a direct consequence of a pair of other propositions.  Another rule of direct consequence will now be presented, \textbf{the rule of distribution}.  We will say that if \(p\), \(q\), and \(r\) are propositions, then the compound proposition \ref{3.item:7} below is a direct consequence of the compound proposition \ref{3.item:6} by distribution.
\begin{enumerate}[resume*]
\item \label{3.item:6} [If \(p\) then [if \(q\) then \(r\)]].
\item \label{3.item:7} [If [if \(p\) then \(q\)] then [if \(p\) then \(r\)]].
\end{enumerate}
Suppose, for example, that \(p\) is the proposition, \(x = 2y\), and \(q\) is the proposition, 3y = z, and \(r\) is the proposition, 3x = 2z.  Then \ref{3.item:6} and \ref{3.item:7} are \ref{3.item:8} and \ref{3.item:9} as follows:
\begin{enumerate}[resume*]
\item \label{3.item:8} [If \(x = 2y\) then [if \(3y = z\) then \(3x = 2z\)]].
\item \label{3.item:9} [If [if \(x = 2y\) then \(3y = z\)] then [if \(x = 2y\) then \(3x = 2z\)]].
\end{enumerate}
Here \ref{3.item:9} is a direct consequence of \ref{3.item:8} by distribution.  In general, a proposition that is a direct consequence of a true proposition by distribution is, itself, true; and a proposition that is a direct consequence of a necessarily true proposition by distribution is, itself, necessarily true.  In particular, \ref{3.item:8} is a necessary truth, so \ref{3.item:9} is also necessarily true.  Even if \(p\), \(q\), and \(r\) in \ref{3.item:6} and \ref{3.item:7} are so chosen that \ref{3.item:6} is not true, it is still the case that \ref{3.item:7} is a direct consequence of \ref{3.item:6} by distribution.

\subsection{}
\label{sec:3.8}

In what follows we shall deal with propositions in very much the way that algebra deals with numbers, not always mentioning them explicitly, but rather representing them by letters.

\section{Formal Proofs}
\label{sec:4}

\subsection{}
\label{sec:4.1}

In logic, as in mathematics, it is desirable to carry out proofs of theorems on the basis of various axioms which are supposed to be logically true.  The axioms can be chosen in various different ways, and also the rules of direct consequence whereby steps are taken from axioms to theorems or from theorems to other theorems.  Some of these differences of choice are extreme enough to give rise to different systems of logic.  Other differences of choice merely represent alternative techniques for formulating one and the same system or equivalent systems.  We say that two systems of logic are \textbf{equivalent} if every theorem of one is a theorem of the other, and vice versa.  A system of logic may be said to be \textbf{valid} if every one of its theorems is a necessarily true proposition.  Two systems of logic, incidentally, could not both be valid and also be such that some theorems of one contradict some theorems of the other.  But two valid systems could differ in the sense that some theorems of one are not theorems of the other.  Results due to Gödel\footnote{K. Godel, “Uber formal unentscheidbare Siitze der Principia Mathematica und verwandter Systeme I”, Monatshefte für Mathematik und Physik, vol.\ 38 (1931), pp.\ 173–98.} seem to indicate that no consistent system, formulated in the usual way, can encompass all logical truths among its theorems.  Thus we can formulate systems of logic that are more and more adequate, but never one that is adequate for all of logical truth.  Indeed, it is often very difficult to know whether or not some complicated proposition is to be regarded as logically true.  The decision often has to be made only in a tentative way and in the light of a careful study of many different systems of logic.

\subsection{}
\label{sec:4.2}

The system of logic of the present book is defined relatively to a set of propositions which we will call the \textbf{axioms} of the system.  The set of axioms will not be completely specified at first, but it has already been partially specified in the sense that we have asserted in \ref{sec:3.5} that every axiom of conditioned repetition is one of our axioms.  The system of logic of the present book is defined also relatively to a set of \textbf{rules of direct consequence}.  These rules, also, will not be completely listed at first, but two of them have already been described, the rule of modus ponens and the rule of distribution.  Finally, the system of logic of the present book is defined relatively to a set of things which we will call \textbf{items}.  For the present we can say that all propositions are items.  Later some things other than propositions will be treated as items.

\subsection{}
\label{sec:4.3}

By a \textbf{formal proof} will be meant a finite sequence of items (usually written as a vertical column or list) such that each item of the sequence satisfies at least one of the following two conditions:
\begin{enumerate}[resume*,start=1]
\item It is an axiom of the system.
\item It is a direct consequence of preceding items of the sequence.
\end{enumerate}

\subsection{}
\label{sec:4.4}

If every axiom is a necessarily true proposition, and if every rule of direct consequence is such that a direct consequence of necessarily true items is itself necessarily true, then clearly each item of a formal proof must be necessarily true.  A formal proof will be said to be a formal proof of each of its items.  In particular, it will be said to be a formal proof of its last item.

\subsection{}
\label{sec:4.5}

An example of a formal proof will be given in \ref{sec:4.6} below.  The letters “\(p\)”, “\(q\)”, “\(r\)” are used, as usual, to stand for any propositions.  The steps of the formal proof are numbered on the left, and a vertical line is drawn between these numbers and the list of items.  To the right of each item is written the “reason” or “justification” for its inclusion in the formal proof, namely, the kind of axiom it is if it is an axiom, or the rule of direct consequence according to which it is a direct consequence of preceding items.  In the latter case the numbers of the relevant preceding items should be stated.

\subsection{}
\label{sec:4.6}

\leavevmode

\noindent
\begingroup
\setlength{\fitchlinewd}{2.909in}
\begin{fitch}
  \fb [If \(p\) then [if \(p\) then \(p\)]] & ax cond rep \\
  \fa [If \(p\) then [if [if \(p\) then \(p\)] then \(p\)]] & ax cond rep \\
  \fa [If [if \(p\) then [if \(p\) then \(p\)]] then [if \(p\) then \(p\)]] & 2, dist \\
  \fa [If \(p\) then \(p\)] & 1, 3, m p
\end{fitch}
\endgroup

\subsection{}
\label{sec:4.7}

Observe that each of the four items of the above formal proof is necessarily true.  In particular, the last item is rather easily recognized as being necessarily true.  The abbreviations “ax cond rep”, “dist”, and “m p” mean, respectively, “axiom of conditioned repetition”, “rule of distribution”, and “rule of modus ponens”.  Item 4, for example, is a direct consequence of items 2 and 3 by the rule of modus ponens.  The formal proof \ref{sec:4.6} is a formal proof of the proposition, [If \(p\) then \(p\)].

\subsection{}
\label{sec:4.8}

The notion of formal proof will now be extended in such a way that we will allow a formal proof to possess items called \textbf{hypotheses}.  A formal proof that has no hypothesis is still correctly described as in \ref{sec:4.3} and will be known as a \textbf{categorical proof}.  Formal proofs that possess one or more hypotheses will be known as \textbf{hypothetical proofs}.  Such a formal proof is, by definition, a finite sequence of items (usually written as a vertical column or list) such that the items which are the hypotheses precede all the others, and such that each item satisfies at least one of the following three conditions:
\begin{enumerate}[resume*,start=1]
\item It is an axiom of the system.
\item It is a direct consequence of preceding items of the sequence.
\item It is a hypothesis of the sequence.
\end{enumerate}
A proposition might appear as a hypothesis of the sequence and also appear subsequently in the sequence.  In its later appearance in the sequence, it could still be said to satisfy condition (3) above.  It will be customary to use the same notation in writing hypothetical proofs as in writing formal proofs that have no hypotheses, except that the hypotheses will be listed at the beginning and will be separated from the subsequent items by a short horizontal line or dash.  After each hypothesis we will write “hyp” to indicate its status.  Any finite number of items may be chosen to serve as the hypotheses of a hypothetical proof, but they must all be carefully listed and designated as such.  It is permissible for a hypothetical proof to have just a single proposition as its hypothesis.  The items of a hypothetical proof will not all be true unless the hypotheses are all true, and the items will not all be necessarily true unless the hypotheses are all necessarily true.  This is on the supposition that all our axioms are necessarily true and that our rules of direct consequence never provide a transition from items that are necessarily true to items that are not necessarily true, or from items that are true to items that are not true.  Some examples of hypothetical proofs will now be given.

\medskip
\subsection{}
\label{sec:4.9}

\begin{fitch}
  \fh \(p\) & hyp \\
  \fa [If \(p\) then [if \(q\) then \(p\)]] & ax cond rep \\
  \fa [If \(q\) then \(p\)] & 1, 2, m p
\end{fitch}

\medskip
\subsection{}
\label{sec:4.10}

\leavevmode

\noindent
\begingroup
\setlength{\fitchlinewd}{2.8in}
\begin{fitch}
  \fh [If \(q\) then \(r\)] & hyp \\
  \fa [If [if \(q\) then \(r\)] then [if \(p\) then [if \(q\) then \(r\)]]] & ax cond rep \\
  \fa [If \(p\) then [if \(q\) then \(r\)]] & 1, 2, m p \\
  \fa [If [if \(p\) then \(q\)] then [if \(p\) then \(r\)]] & 3, dist
\end{fitch}
\endgroup

\medskip
\subsection{}
\label{sec:4.11}

For convenience and brevity, a more concise notation will now be introduced.  An expression of the form, “[If \(p\) then \(q\)]”, will hereafter be written as “[\(p \supset q\)]”.  The horseshoe symbol can be read as “implies”, but a more accurate reading is the \emph{if-then} reading.  Thus the proposition, [(It is raining) \(\supset\) (the ground is wet)], can be read as, “If it is raining, then the ground is wet”, or less accurately but often more conveniently as, “It is raining implies that the ground is wet”.  Using the horseshoe notation, we now rewrite \ref{sec:4.6}, \ref{sec:4.9}, \ref{sec:4.10}, respectively, as \ref{sec:4.12}, \ref{sec:4.13}, \ref{sec:4.14}.  It will be customary to omit the outermost pair of square brackets from each item of a formal proof.  In general, when propositions are displayed separately or are written in a list, we omit outermost square brackets.  We will call \ref{sec:4.12} the principle of \textbf{reflexivity of implication} (“refl imp”).  Strictly speaking, this principle is illustrated by the last step of \ref{sec:4.12} rather than by the whole of \ref{sec:4.12}.

\subsection{}
\label{sec:4.12}

\begin{fitch}
  \fb \(p \supset [p \supset p]\) & ax cond rep \\
  \fa \(p \supset [[p \supset p] \supset p]\) & ax cond rep \\
  \fa \([p \supset [p \supset p]] \supset [p \supset p]\) & 2, dist \\
  \fa \(p \supset p\) & 1, 3, m p
\end{fitch}

\smallskip
\subsection{}
\label{sec:4.13}

\begin{fitch}
  \fh \(p\) & hyp \\
  \fa \(p \supset [q \supset p]\) & ax cond rep \\
  \fa \(q \supset p\) & 1, 2, m p
\end{fitch}

\smallskip
\subsection{}
\label{sec:4.14}

\begin{fitch}
  \fh \(q \supset r\) & hyp \\
  \fa \([q \supset r] \supset [p \supset [q \supset r]]\) & ax cond rep \\
  \fa \(p \supset [q \supset r]\) & 1, 2, m p \\
  \fa \([p \supset q] \supset [p \supset r]\) & 3, dist
\end{fitch}

\smallskip
\subsection{}
\label{sec:4.15}

We will say that \ref{sec:4.13} is a hypothetical proof of [\(q \supset p\)] on the hypothesis \(p\), and we will say that \ref{sec:4.14} is a hypothetical proof of [\([p \supset q] \supset [p \supset r]\)] on the hypothesis [\(q \supset r\)].  A similar way of speaking will be used with respect to other hypothetical proofs.

% \medskip
\subsection{}
\label{sec:4.16}

Observe that steps 1 through 3 of \ref{sec:4.14} are exactly like steps 1 through 3 of \ref{sec:4.13} except that [\(q \supset r\)] and \(p\), respectively, take the places of \(p\) and \(q\).  Since these steps are just the same (with the exception noted), it would be more convenient in \ref{sec:4.14} simply to refer back to \ref{sec:4.13} instead of repeating something which has, in effect, already been done in \ref{sec:4.13}.  Thus \ref{sec:4.14} could be written more briefly as follows:

% \medskip
\subsection{}
\label{sec:4.17}

\begingroup
\setlength{\fitchlinewd}{2.4in}
\begin{fitch}
  \fh \(q \supset r\) & hyp \\
  % \fa \([q \supset r] \supset [p \supset [q \supset r]]\) & ax cond rep \\
  \fa \(p \supset [q \supset r]\) & 1, \ref{sec:4.13} \\
  \fa \([p \supset q] \supset [p \supset r]\) & 3, dist
\end{fitch}
\endgroup

\smallskip
\subsection{}
\label{sec:4.18}

Observe that the notation “1, \ref{sec:4.13}”, at the right of step 2 in \ref{sec:4.17}, indicates that step 2 is obtainable from step 1 by the same method used to obtain the last step of \ref{sec:4.13} from the hypothesis of \ref{sec:4.13}.  In other words, there are missing steps in \ref{sec:4.17} which the reader can supply by referring to \ref{sec:4.13}.  Actually \ref{sec:4.17} is not, as written, a formal proof in the sense of \ref{sec:4.8}, unless the missing steps are inserted.  We can think of \ref{sec:4.17}, as written, as being an abbreviation for the full formal proof given in \ref{sec:4.14}.

% \smallskip
\subsection{}
\label{sec:4.19}

A very useful hypothetical proof will next be presented.  It is seen to possess two hypotheses.  The formal proof will first be presented in full as \ref{sec:4.20}, and then in abbreviated form as \ref{sec:4.21}.  This formal proof will be called the principle of the \textbf{transitivity of implication} (“trans imp”).  The reasons for the various steps in \ref{sec:4.20} and \ref{sec:4.21} are purposely omitted so that the reader may insert them as an exercise.

% \smallskip
\subsection{}
\label{sec:4.20}

\begin{fitch}
  \fb \(p \supset q\) & % hyp
  \\
  \fj \(q \supset r\) & % hyp
  \\
  \fa \([q \supset r] \supset [p \supset [q \supset r]]\) & % ax cond rep
  \\
  \fa \(p \supset [q \supset r]\) & % 2, 3, m p
  \\
  \fa \([p \supset q] \supset [p \supset r]\) & % 4, dist
  \\
  \fa \(p \supset r\) & % 1, 4, m p
\end{fitch}

% \smallskip
\subsection{}
\label{sec:4.21}

\begin{fitch}
  \fb \(p \supset q\) & % hyp
  \\
  \fj \(q \supset r\) & % hyp
  \\
  \fa \([p \supset q] \supset [p \supset r]\) & % 2, \ref{sec:4.14}
  \\
  \fa \(p \supset r\) & % 1, 3, m p
\end{fitch}

% \smallskip
\subsection{}
\label{sec:4.22}

Observe that the order of hypotheses does not make any difference.  Thus \ref{sec:4.21} could equally well have been written as follows:

% \smallskip
\subsection{}
\label{sec:4.23}

\begin{fitch}
  \fb \(q \supset r\) & % hyp
  \\
  \fj \(p \supset q\) & % hyp
  \\
  \fa \([p \supset q] \supset [p \supset r]\) & % 2, \ref{sec:4.17}
  \\
  \fa \(p \supset r\) & % 1, 3, m p
\end{fitch}

% \smallskip
\subsection{}
\label{sec:4.24}

Another useful hypothetical proof will next be presented.  It will be called the principle of \textbf{conditioned modus ponens} (“cond m p”).

% \smallskip
\subsection{}
\label{sec:4.25}

\begin{fitch}
  \fb \(s \supset p\) & hyp \\
  \fj \(s \supset [p \supset q]\) & hyp \\
  \fa \([s \supset p] \supset [s \supset q]\) & 2, dist \\
  \fa \(s \supset q\) & 1, 3, m p
\end{fitch}

\subsection{}
\label{sec:4.26}

In addition to those propositions which are axioms of conditioned repetition, some further propositions will now be counted as being axioms of this system.  These further propositions are all those of the form, [\(p_1 \supset p_2\)], where \(p_1\) and \(p_2\) have so been chosen that \(p_2\) is a direct consequence of \(p_1\) by distribution.  Such axioms will be called \textbf{distributive axioms} (“ax dist”).  Thus every proposition of the form, [\([p \supset [q \supset r]] \supset [[p \supset q] \supset [p \supset r]]\)], is a distributive axiom.  Such an axiom is used in step 2 of \ref{sec:4.27} below.  We will call \ref{sec:4.27} the principle of \textbf{conditioned distribution} (“cond dist”).

\subsection{}
\label{sec:4.27}

\leavevmode

\noindent
\begingroup
\setlength{\fitchlinewd}{2.328in}
\begin{fitch}
  \fh \(s \supset [p \supset [q \supset r]]\) & hyp \\
  \fa \([p \supset [q \supset r]] \supset [[p \supset q] \supset [p \supset r]]\) & ax dist \\
  \fa \( s \supset [[p \supset q] \supset [p \supset r]]\) & 1, 2, trans imp (\ref{sec:4.20})
\end{fitch}
\endgroup

\subsection{}
\label{sec:4.28}

If we wished, we could now dispense entirely with the rule of distribution, since the same results can always be obtained by using the requisite distributive axiom together with the rule of modus ponens.  Thus we can make the transition from [\(p \supset [q \supset r]\)] to [\([p \supset q] \supset [p \supset r]\)] as in \ref{sec:4.29}.

\subsection{}
\label{sec:4.29}

\begingroup
\setlength{\fitchlinewd}{2.175in}
\begin{fitch}
  \fh \(p \supset [q \supset r]\) & hyp \\
  \fa \([p \supset [q \supset r]] \supset [[p \supset q] \supset [p \supset r]]\) & ax dist \\
  \fa \([p \supset q] \supset [p \supset r]\) & 1, 2, m p
\end{fitch}
\endgroup

\subsection{}
\label{sec:4.30}

The two following categorical proofs will be referred to subsequently in 5.21.

\subsection{}
\label{sec:4.31}

\setlength{\fitchlinewd}{2.416in}
\begin{fitch}
  \fb \(p \supset [q \supset r]\) & ax cond rep \\
  \fa \(s \supset [p \supset [q \supset r]]\) & 1, \ref{sec:4.13}
\end{fitch}

\subsection{}
\label{sec:4.32}

\begin{fitch}
  \fb \([p \supset [q \supset r]] \supset [[p \supset q] \supset [p \supset r]]\) & ax dist \\
  \fa \(s \supset [p \supset [q \supset r]] \supset [[p \supset q] \supset [p \supset r]]\) & 1, \ref{sec:4.13}
\end{fitch}

\subsection{}
\label{sec:4.33}

The theorems of the present system of logic are all those propositions for which there are categorical proofs.  For example, every proposition of the form, [\(p \supset p\)], is a theorem of this system, and so is every proposition of the form, [\([s \supset p] \supset [s \supset [r \supset r]]\)].

\section*{Exercises}

\begin{enumerate}[left=\parindent..0pt,itemindent=*]
\item Give a categorical proof of [\(p \supset [q \supset q]\)].
\item Give a categorical proof of [\([s \supset p] \supset [s \supset [r \supset r]]\)].
\item Give a hypothetical proof of [\(q \supset [r \supset p]\)] on the hypothesis \(p\).
\item Give a hypothetical proof of [\(p \supset r\)] on the hypotheses \(q\) and [\(p \supset [q \supset r]\)].
\item Give a hypothetical proof of [\(s \supset [r \supset p]\)] on the hypothesis [\(s \supset p\)].
\item Give a hypothetical proof of [\(p \supset q\)] on the hypothesis [\(p \supset [p \supset q]\)].
\item Give a hypothetical proof of [\(q \supset [p \supset r]\)] on the hypothesis [\(p \supset [q \supset r]\)].  (This exercise is more difficult than the others.)
\end{enumerate}

\openrightfalse
\chapter{The Method of Subordinate Proofs}
\label{chap:2}

\end{document}

% Local Variables:
% TeX-engine: luatex
% End:
